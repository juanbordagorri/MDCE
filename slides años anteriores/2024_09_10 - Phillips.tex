\documentclass[11pt]{beamer}
\usepackage[utf8]{inputenc}
\usepackage[T1]{fontenc}
%\usepackage{natbib}
\usetheme{Pittsburgh}
\usepackage{verbatim} 
\usepackage[english]{babel}
\usepackage{epstopdf}
\usepackage{multicol}
%\titlegraphic{%\vspace*{1cm}
	%	\includegraphics[width=2.5cm]{logo_udelar}
	%\hspace*{1cm}~%
	%		\includegraphics[width=3.5cm]{logo_FCEA.png}
	%}
\setbeamertemplate{navigation symbols}{}
\setbeamertemplate{footline}[frame number]
\AtBeginSection{ 
	\begin{frame}
		\frametitle{Index}
		\tableofcontents[currentsection]
	\end{frame}
}
\begin{document}
	\title{Modelos dinámicos y computacionales en Economía}
	\subtitle{Curva de Phillips}
	%\logo{}
	\institute{FCEA, UDELAR}
	\date{10 de septiembre de 2024}
	%\subject{}
	%\setbeamercovered{transparent}
	%\setbeamertemplate{navigation symbols}{}
	\frame[plain]{\maketitle}
	%\setbeamertemplate{background}{\includegraphics[width=2 cm]{logo_FCEA.png}}
	
	
	
	\begin{frame}{Introducción}
		Es un modelo macro utilizado para analizar la dinámica de la inflación y la brecha del producto.\\
		Tres casos:\\
\begin{enumerate}
	\item control de la inflación
	\item perturbación permanente de demanda
	\item perturbación temporal de oferta
\end{enumerate}
		Estratégias de control:\\
\begin{enumerate}
	\item agresiva
	\item gradual
\end{enumerate}		
		
		Tres tipos de respuestas de los hacedores de politica:
\begin{enumerate}
	\item neutral
	\item complaciente
	\item de extinción
\end{enumerate}		
	\end{frame}


 \begin{frame}{Introducción}
 \framesubtitle{Relación entre inflación y brecha de producto. \\Fuente: Gordon (2009)}
     \begin{figure}
         \centering
         \includegraphics[width=0.6\linewidth]{images/Captura desde 2024-09-10 15-45-46.png}
   %      \caption{Caption}
         \label{fig:enter-label}
     \end{figure}
 \end{frame}
	\begin{frame}{Tres ecuaciones centrales}
		(1) Formación de expectativas:
		
		$$
		\pi_{t}^{e}=\lambda \pi_{t-1}+(1-\lambda) \pi_{t-1}^{e}, \lambda \in[0,1]
		$$
		
		(2) Oferta agregada de corto plazo:
		
		$$
		\pi_{t}=\pi_{t}^{e}+\alpha \widehat{Y}_{t}+z_{t}^{s}
		$$
		
		(3) Demanda agregada de corto plazo:
		
		$$
		\widehat{Y}_{t}=\widehat{x}_{t}-\pi_{t}+\widehat{Y}_{t-1}+z_{t}^{d}
		$$
		
	\end{frame}
	
	
	\begin{frame}{Variables endógenas}
		\begin{itemize}
			\item inflación efectiva en el período $t: \pi_{t}$
			\item inflación esperada: $\pi_{t}^{e}$
			\item desvio porcentual entre el producto efectivo y el producto natural: $\widehat{Y}_{t}$
		\end{itemize}
		
	\end{frame}

		
\begin{frame}[allowframebreaks]
			\frametitle{Variables exógenas}
			\begin{itemize}
				\item Shock de oferta sobre la inflación $\left(z_{t}^{s}\right)$ y shock de demanda $\left(z_{t}^{d}\right)$
				\item Inflación efectiva y esperada en el período anterior: $\pi_{t-1}$ y $\pi_{t-1}^{e}$\\
				parámetros: $\lambda, \alpha$\\
				otras variables requeridas: $Y_{t}$ producto real, $X_{t}$ producto nominal, $\widehat{x}_{t}$ tasa de crec. del prod. nominal y $P_{t}$ nivel general de precios.
			\end{itemize}
		\end{frame}
  \begin{frame}[allowframebreaks]
  \frametitle{Proceso de formación de expectativas}
      $\pi_{t}^{e}=\lambda \pi_{t-1}+(1-\lambda) \pi_{t-1}^{e}$\\
			Casos extremos:
			
			\begin{itemize}
				\item $\lambda=0$ los agentes no ajustan expectativas $\Rightarrow \pi_{t}^{e}=\pi_{t-1}^{e}$
				\item $\lambda=1$ expectativas adaptativas con velocidad de ajuste máximo, $\pi_{t}^{e}=\pi_{t-1}$
			\end{itemize}
			
			Observar que:\\
			si $\pi_{t}^{e}=\lambda \pi_{t-1}+(1-\lambda) \pi_{t-1}^{e} \Rightarrow \pi_{t-1}^{e}=\lambda \pi_{t-2}+(1-\lambda) \pi_{t-2}^{e}$\\
			$\mathrm{y} \pi_{t}^{e}=\lambda \pi_{t-1}+(1-\lambda)\left[\lambda \pi_{t-2}+(1-\lambda) \pi_{t-2}^{e}\right]$
			
			$$
			\pi_{t}^{e}=\lambda \pi_{t-1}+\lambda(1-\lambda) \pi_{t-2}+(1-\lambda)^{2} \pi_{t-2}^{e}
			$$
			
			$$
			\pi_{t}^{e}=\lambda\left[\pi_{t-1}+(1-\lambda) \pi_{t-2}+(1-\lambda)^{2} \pi_{t-3}+\ldots\right]
			$$
	\end{frame}
 \begin{frame}{Proceso de formación de expectativas (cont.)}
     Y se cumple que:
			
			$$
			\lambda\left[1+(1-\lambda)+(1-\lambda)^{2}+(1-\lambda)^{3}+\cdots \cdots\right]=1
			$$
			
			Esto es: la $\pi_{t}^{e}$ es un promedio ponderado de la inflación observada en todos los períodos anteriores, donde los ponderadores van perdiendo peso cuanto mas lejanos en el tiempo se encuentren.
			
 \end{frame}
			
		
\begin{frame}{Curva SP: curva de Phillips aumentada con expectativas de corto plazo}
	$$
	\pi_{t}=\pi_{t}^{e}+\alpha \widehat{Y}_{t}+z_{t}^{s}
	$$
	
	En el equilibrio de largo plazo la curva en el plano $\left(\widehat{Y}_{t}, \pi_{t}\right)$ esta representada por una recta vertical.\\
	A partir de:
	
	\[
	\left\{\begin{array}{l}
		\pi_{t}^{e}=\lambda \pi_{t-1}+(1-\lambda) \pi_{t-1}^{e}  \tag{1}\\
		\pi_{t}=\pi_{t}^{e}+\alpha \widehat{Y}_{t}+z_{t}^{s}
	\end{array}\right.
	\]
	
	Se llega a:
	
	
	\begin{equation*}
		\pi_{t}=\lambda \pi_{t-1}+(1-\lambda) \pi_{t-1}^{e}+\alpha \widehat{Y}_{t}+z_{t}^{s} \tag{2}
	\end{equation*}
	
\end{frame}
\begin{frame}{Curva DG: demanda agregada}

\begin{gather*}
X_{t}=P_{t} Y_{t} \\
\log X_{t}-\log X_{t-1}=\log P_{t}-\log P_{t-1}+\log Y_{t}-\log Y_{t-1} \\
x_{t}=\pi_{t}+y_{t} \tag{3}
\end{gather*}


Tasa de crecimiento del producto nominal $=$ inflación + tasa del crecimiento del prod. real

$$
\begin{aligned}
& y_{t}^{N}=\log Y_{t}^{N}-\log Y_{t-1}^{N} \\
& x_{t}-y_{t}^{N}=\pi_{t}+y_{t}-y_{t}^{N}
\end{aligned}
$$

exceso de crecimiento producto nominal $=$ inflación + exceso de crecimiento del producto real respecto al natural.
\end{frame}

\begin{frame}{Ejemplo: ajuste de inflación y brecha de producto}
\framesubtitle{Ajuste de los precios ante un aumento del producto nominal del 6\%}
    \begin{figure}
        \centering
        \includegraphics[width=0.65\linewidth]{images/Captura desde 2024-09-10 16-18-53.png}
  %      \caption{Caption}
        \label{fig:enter-label}
    \end{figure}
\end{frame}


\begin{frame}{Ejemplo: ajuste de inflación y brecha de producto (cont.)}
\framesubtitle{Ajuste de los precios ante un aumento del producto nominal del 6\%}
\begin{figure}
    \centering
    \includegraphics[width=0.65\linewidth]{images/Captura desde 2024-09-10 16-38-22.png}
%    \caption{Caption}
    \label{fig:enter-label}
\end{figure}
\end{frame}
\begin{frame}
Y se puede re escribir como:


\begin{equation*}
\widehat{Y}_{t}=\widehat{x}_{t}-\pi_{t}+\widehat{Y}_{t-1}+z_{t}^{d} \tag{4}
\end{equation*}


Representa los pares $\left(\widehat{Y_{t}}, \pi_{t}\right)$ para una tasa dada de crecimiento de la demanda $\left(\widehat{x}_{t}\right)$		
		

		
\end{frame}		

		
\begin{frame}{Equilibrio Global}
	Se corresponde con el par $\left(\widehat{Y}_{t}, \pi_{t}\right)$ que se verifica SP y DG.\\
	Sustituyendo
	
	$$
	\widehat{Y_{t}}=\widehat{x}_{t}-\pi_{t}+\widehat{Y}_{t-1}+z_{t}^{d}
	$$
	
	en
	
	$$
	\pi_{t}=\lambda \pi_{t-1}+(1-\lambda) \pi_{t-1}^{e}+\alpha \widehat{Y}_{t}+z_{t}^{s}
	$$
	
	Analiticamente se corresponde con:
	
	
	\begin{equation*}
		\pi_{t}=\left(\frac{1}{1+\alpha}\right)\left[\lambda \pi_{t-1}+(1-\lambda) \pi_{t-1}^{e}+\alpha\left(\widehat{x}_{t}+\widehat{Y}_{t-1}-z_{t}^{d}\right)+z_{t}^{s}\right] \tag{5}
	\end{equation*}
	

\end{frame}


\begin{frame}
		Sustituyendo en:


\begin{equation*}
	\widehat{Y}_{t}=\widehat{x}_{t}-\pi_{t}+\widehat{Y}_{t-1}+z_{t}^{d} \tag{6}
\end{equation*}


la última ecuación, se llega a la ecuación que describe la dinámica del producto:

$$
\begin{gathered}
	\widehat{Y}_{t}=\widehat{x}_{t}-\frac{1}{1+\alpha}\left[\pi_{t-1}+(1-\lambda) \pi_{t-1}^{e}+\alpha \widehat{Y}_{t-1}\right] \\
	-\frac{\alpha}{1+\alpha}\left(\widehat{x}_{t}+z_{t}^{d}\right)-\frac{1}{1+\alpha} z_{t}^{s}+Y_{t-1}+z_{t}^{d}(7)
\end{gathered}
$$
	
\end{frame}	

\begin{frame}
		Otra forma de analizarlo\\
Considerar (la forma canonica del sistema):

\[
\left\{\begin{array}{l}
	\pi_{t}=\frac{1}{1+\alpha}\left(\lambda \pi_{t-1}+(1-\lambda) \pi_{t-1}^{e}+\alpha \widehat{Y}_{t-1}\right)+K  \tag{8}\\
	\pi_{t}^{e}=\frac{1}{1+\alpha}\left(\lambda \pi_{t-1}+(1-\lambda) \pi_{t-1}^{e}\right) \\
	\widehat{Y}_{t}=\frac{1}{1+\alpha}\left(-\lambda \pi_{t-1}+(1-\lambda) \pi_{t-1}^{e}+\alpha \widehat{Y}_{t-1}\right)
\end{array}\right.
\]	
\end{frame}

		
\begin{frame}
		En forma matricial:
	
	
	\begin{gather*}
		\left(\begin{array}{c}
			\pi_{t} \\
			\pi_{t}^{e} \\
			\widehat{Y}_{t}
		\end{array}\right)=\left(\begin{array}{ccc}
			\lambda & 1-\lambda & \alpha \\
			\lambda & 1-\lambda & 0 \\
			-\lambda & 1-\lambda & \alpha
		\end{array}\right)\left(\begin{array}{c}
			\pi_{t-1} \\
			\pi_{t-1}^{e} \\
			\frac{Y_{t-1}}{e}
		\end{array}\right) \\
		+\left(\begin{array}{l}
			\frac{\alpha}{1+\alpha}\left(\widehat{x}_{t}+z_{t}^{d}\right)+z_{t}^{s} \\
			0 \\
			\left.\frac{1}{1+\alpha}\left(\widehat{x}_{t}+z_{t}^{d}\right)-z_{t}^{s}\right)
		\end{array}\right) \tag{9}
	\end{gather*}
\end{frame}		
	

	
\end{document}