
\documentclass{article}
%%%%%%%%%%%%%%%%%%%%%%%%%%%%%%%%%%%%%%%%%%%%%%%%%%%%%%%%%%%%%%%%%%%%%%%%%%%%%%%%%%%%%%%%%%%%%%%%%%%%%%%%%%%%%%%%%%%%%%%%%%%%%%%%%%%%%%%%%%%%%%%%%%%%%%%%%%%%%%%%%%%%%%%%%%%%%%%%%%%%%%%%%%%%%%%%%%%%%%%%%%%%%%%%%%%%%%%%%%%%%%%%%%%%%%%%%%%%%%%%%%%%%%%%%%%%
\usepackage{amsfonts}
\usepackage{amsmath}

\setcounter{MaxMatrixCols}{10}
%TCIDATA{OutputFilter=LATEX.DLL}
%TCIDATA{Version=5.50.0.2953}
%TCIDATA{<META NAME="SaveForMode" CONTENT="1">}
%TCIDATA{BibliographyScheme=Manual}
%TCIDATA{Created=Thursday, October 14, 2004 17:42:35}
%TCIDATA{LastRevised=Monday, July 22, 2024 17:04:30}
%TCIDATA{<META NAME="GraphicsSave" CONTENT="32">}
%TCIDATA{<META NAME="DocumentShell" CONTENT="Standard LaTeX\Blank - Standard LaTeX Article">}
%TCIDATA{CSTFile=40 LaTeX article.cst}

\newtheorem{theorem}{Theorem}
\newtheorem{acknowledgement}[theorem]{Acknowledgement}
\newtheorem{algorithm}[theorem]{Algorithm}
\newtheorem{axiom}[theorem]{Axiom}
\newtheorem{case}[theorem]{Case}
\newtheorem{claim}[theorem]{Claim}
\newtheorem{conclusion}[theorem]{Conclusion}
\newtheorem{condition}[theorem]{Condition}
\newtheorem{conjecture}[theorem]{Conjecture}
\newtheorem{corollary}[theorem]{Corollary}
\newtheorem{criterion}[theorem]{Criterion}
\newtheorem{definition}[theorem]{Definition}
\newtheorem{example}[theorem]{Example}
\newtheorem{exercise}[theorem]{Exercise}
\newtheorem{lemma}[theorem]{Lemma}
\newtheorem{notation}[theorem]{Notation}
\newtheorem{problem}[theorem]{Problem}
\newtheorem{proposition}[theorem]{Proposition}
\newtheorem{remark}[theorem]{Remark}
\newtheorem{solution}[theorem]{Solution}
\newtheorem{summary}[theorem]{Summary}
\newenvironment{proof}[1][Proof]{\noindent\textbf{#1.} }{\ \rule{0.5em}{0.5em}}
\input{tcilatex}
\begin{document}

\title{Ejercicios de ecuaciones en diferencias y sistemas din\'{a}micos
discretos. (1)}
\author{GIDE}
\date{UDELAR\ 2024}
\maketitle

\begin{enumerate}
\item Exprese $a_{n}$ en funci\'{o}n de los t\'{e}rminos anteriores ($a_{k}$
con $k\leq n-1$) y resuelva la relativa ecuaci\'{o}n en diferencias
correspondiente, siendo $a_{n}$:

\begin{enumerate}
\item El monto de una cuenta bancaria al $n-$\'{e}simo mes de haber sido
abierta si se paga un inter\'{e}s del $i$\% mensual y el due\~{n}o retira $r$
euro por mes.

\item El n\'{u}mero de puntos de cortes de $n$ l\'{\i}neas diferentes no
paralelas ni concurrentes del plano.

\item El n\'{u}mero de secuencias de ceros y unos de largo $n$ en las cuales
no aparecen dos ceros seguidos.

\item El n\'{u}mero de secuencias de ceros y unos de largo $n$ en las cuales
no aparecen dos ceros ni dos unos seguidos.

\item La cantidad de saludos que se dieron los primeros $n$ invitados de una
reuni\'{o}n, si cada vez que llego uno, \'{e}ste salud\'{o} el resto.
\end{enumerate}

\item Supongamos que la suma constante $T$ se deposita al final de cada per%
\'{\i}odo fijo en un banco que paga una tasa de inter\'{e}s $r$ por per\'{\i}%
odo. Sea $a_{n}$ la cantidad acumulada en el banco luego de $n$ per\'{\i}%
odos.

\begin{enumerate}
\item Escribir la ecuaci\'{o}n en diferencias que describe $a_{n}$.

\item Resolver la ecuaci\'{o}n.
\end{enumerate}

\item Se considera la ecuaci\'{o}n en diferencias 
\begin{equation*}
x_{n+1}=\alpha x_{n}
\end{equation*}%
A partir de un valor inicial $x_{0}$ dado, esbozar el gr\'{a}fico de la
telara\~{n}a $x_{n}$ en los casos: $1<\alpha <0,\alpha <1$ y $\alpha =1$ y a
partir de esto calcular $\underset{n\rightarrow +\infty }{\lim }x_{n}$
dicutiendo seg\'{u}n el valor de $\alpha $. Idem para la ecuaci\'{o}n%
\begin{equation*}
x_{n+1}=\alpha x_{n}+\beta
\end{equation*}

\item Encuentre la soluci\'{o}n general de las siguientes ecuaciones:

\begin{enumerate}
\item $a_{n+1}=\frac{3}{2}a_{n};a_{0}=1$

\item $a_{n+1}=na_{n};a_{0}=a$

\item $a_{n+1}=\frac{n}{n+1}a_{n};n>0$

\item $a_{n+1}+a_{n}=2n+3;a_{0}=1$

\item $a_{n+1}-2a_{n}=3n^{2}-n;a_{0}=3$

\item $a_{n+1}-2a_{n}=6;a_{0}=1$

\item $a_{n+1}-4a_{n}=2^{n};a_{0}=-1$
\end{enumerate}

\item \textbf{Ecuaciones no lineales transformables a ecuaciones lineales}

\begin{enumerate}
\item Ecuaci\'{o}n de Riccati: Resolver la ecuaci\'{o}n%
\begin{equation*}
x_{n+1}x_{n}+px_{n+1}+qx_{n}=0
\end{equation*}%
mediante el cambio de variable%
\begin{equation*}
y_{n}=\frac{1}{x_{n}}
\end{equation*}

\item Generalizaci\'{o}n: Resolver la ecuaci\'{o}n%
\begin{equation*}
x_{n+1}=\frac{ax_{n}+b}{cx_{n}+d}
\end{equation*}%
mediante el cambio de variable%
\begin{equation*}
\frac{y_{n+1}}{y_{n}}=cx_{n}+d
\end{equation*}

\item Resolver la ecuaci\'{o}n%
\begin{equation*}
x_{n+1}=\frac{2x_{n}+3}{3x_{n}+2}
\end{equation*}%
y estudiar directamente la estabilidad de los equilibrios.

\item Resolver la ecuaci\'{o}n%
\begin{equation*}
x_{n+1}=x_{n}^{2}
\end{equation*}%
y calcular $\underset{n\rightarrow +\infty }{\lim }x_{n}$.

\item Resolver la ecuaci\'{o}n%
\begin{equation*}
x_{n+1}=2x_{n}\left( 1-x_{n}\right)
\end{equation*}%
y calcular $\underset{n\rightarrow +\infty }{\lim }x_{n}$ dicutiendo seg\'{u}%
n el valor de la condici\'{o}n inicial. Sugerencia: usar el cambio de
variable $y_{n}=1-2x_{n}$ y la ecuaci\'{o}n precedente.
\end{enumerate}

\item En el modelo de oferta y demanda supongamos que la curva de oferta
esta dada por%
\begin{equation*}
S_{n+1}=p_{n}^{2}+1
\end{equation*}%
y la curva de demanda esta dada por%
\begin{equation*}
D_{n}=-2p_{n}+3
\end{equation*}

\begin{enumerate}
\item Hallar la ecuaci\'{o}n en diferencias que relaciona $p_{n+1}$ con $%
p_{n}$.

\item Hallar el precio de mercado $p^{\ast }$ y estudiar su estabilidad.
\end{enumerate}

\item Consideremos la siguiente extensi\'{o}n del modelo de la telara\~{n}a 
\begin{equation*}
\begin{array}{c}
D_{n}=b_{D}-m_{D}p_{n} \\ 
S_{n}=b_{S}+m_{S}p_{n}^{e} \\ 
D_{n}=S_{n} \\ 
p_{n}^{e}=p_{n-1}^{e}+\lambda \left( p_{n-1}-p_{n-1}^{e}\right)%
\end{array}%
\end{equation*}%
donde $0<\lambda <1$.

\begin{enumerate}
\item Mostrar que el precio $p_{n}$ sigue una ecuaci\'{o}n en diferencias
lineal de primer orden no homog\'{e}nea.

\item Obtener el equilibrio y mostrar que es asint\'{o}ticamente estable si $%
0<\lambda <\frac{2m_{D}}{m_{D}+m_{S}}$.
\end{enumerate}

\item Esbozar el gr\'{a}fico de $y=2x^{2}-1$ y considerando el m\'{e}todo de
la telara\~{n}a para la ecuaci\'{o}n en diferencias $x_{n+1}=2x_{n}^{2}-1$,
calcular $\underset{n\rightarrow +\infty }{\lim }x_{n}$ dicutiendo seg\'{u}n
el valor de la condici\'{o}n inicial.

\item Estudiar el comportamiento de las soluciones de la ecuaci\'{o}n en
diferencias:

\begin{enumerate}
\item $x_{n+1}=e^{x_{n}}$

\item $x_{n+1}=\sqrt{4x_{n}-3}$

\item $x_{n+1}=x_{n}^{3}-x_{n}^{2}+1$
\end{enumerate}

\item Sea $f(x)=\frac{3x-1}{x+1}.$

\begin{enumerate}
\item Estudiar el signo de $f(x)-x$.

\item Se considera la ecuaci\'{o}n en diferencias $x_{n+1}=f(x_{n}).$
Determinar los equilibrios y estudiar su estabilidad.

\item Si $x_{0}=2,$ calcular $\underset{n\rightarrow +\infty }{\lim }x_{n}$.

\item Si $x_{0}=\frac{1}{2},$ determinar si $x_{n}$ puede tomar valores
negativos.
\end{enumerate}

\item \textbf{Ecuaci\'{o}n log\'{\i}stica de Pielou} (E.C. Pielou, \textit{%
An introduction to Mathematical Ecology}, Wiley Interscience, New York,
1969) E.C. Pielou se refiere a la siguiente ecuaci\'{o}n como el equivalente
discreto de la ecuaci\'{o}n log\'{\i}stica:%
\begin{equation*}
x_{n+1}=\frac{\alpha x_{n}}{1+\beta x_{n}}
\end{equation*}%
donde $\alpha >1$ y $\beta >0$.

\begin{enumerate}
\item Hallar el equilibrio positivo $p$.

\item Tomando $\alpha =2$ y $\beta =1$, mostrar que el equilibrio $p$ es un
atractor.

\item Discutir la estabilidad de $p$ seg\'{u}n los valores de $\alpha $ y $%
\beta $.

\item Resolver la ecuaci\'{o}n de Pielou y obtener nuevamente los resultados
de las partes anteriores. (Sugerencia: observar que es una ecuaci\'{o}n del
tipo Riccati y usar el relativo cambio de variable).
\end{enumerate}

\item Para las siguientes ecuaciones, hallar los puntos de equilibrio y
discutir su estabilidad.

\begin{enumerate}
\item $x_{n+1}=5-\frac{6}{x_{n}}$

\item $x_{n+1}=x_{n}^{2}+3x_{n}$

\item $x_{n+1}=x_{n}^{2}+\frac{3}{16}$

\item $x_{n+1}=\frac{1}{2}\left( x_{n}^{3}+x_{n}\right) $

\item $x_{n+1}=-\left( x_{n}^{3}+x_{n}\right) $

\item $x_{n+1}=x_{n}^{3}+x_{n}^{2}+x_{n}$

\item $x_{n+1}=x_{n}^{3}-x_{n}^{2}+x_{n}$
\end{enumerate}

\item Mostrar que $0$ es un equilibrio asint\'{o}ticamente estable de la
ecuaci\'{o}n $x_{n+1}=\frac{n+1}{n+2}x_{n}.$

\item Mostrar que $0$ es un equilibrio asint\'{o}ticamente estable de la
ecuaci\'{o}n $x_{n+1}=a_{n}x_{n}$ si y solo si 
\begin{equation*}
\underset{n\rightarrow +\infty }{\lim }\left\vert \underset{i=0}{\overset{n-1%
}{\prod }}a_{i}\right\vert =0
\end{equation*}

\item Mostrar que $\left\{ -1,1\right\} $ es una \'{o}rbita peri\'{o}dica
asint\'{o}ticamente estable del mapa $f(x)=-\frac{1}{2}x^{2}-x+\frac{1}{2}$.

\item Para las siguientes ecuaciones, hallar los equilibrios y los $2-$%
ciclos y discutir su estabilidad.

\begin{enumerate}
\item $x_{n+1}=x_{n}^{2}+x_{n}-4$

\item $x_{n+1}=1-x_{n}^{2}$
\end{enumerate}

\item Hallar los valores de $a,b\in 
%TCIMACRO{\U{211d} }%
%BeginExpansion
\mathbb{R}
%EndExpansion
$ tales que $\left\{ 0,1\right\} $ es una \'{o}rbita peri\'{o}dica asint\'{o}%
ticamente estable del mapa $f(x)=ax^{3}-bx+1.$

\item Puede una funci\'{o}n decreciente tener ciclos? Y una funci\'{o}n
creciente?

\item Se considera el mapa definido por:%
\begin{equation*}
B(x)=\left\{ 
\begin{array}{c}
2x\text{, si }x\in \left[ 0,\frac{1}{2}\right] \\ 
2x-1\text{, si }x\in \left( \frac{1}{2},1\right]%
\end{array}%
\right\vert
\end{equation*}

\begin{enumerate}
\item Graficar la funci\'{o}n $B$.

\item Mostrar que $x\in \left[ 0,1\right] $ es un punto eventualmente fijo
de $B$ sii es de la forma $x=\frac{k}{2^{n}}$ con $k=1,2,...2^{n}-1$.

\item Graficar la funci\'{o}n $B^{2}$, calcular los $2-$ciclos de $B$ y
estudiar su estabilidad.

\item Calcular el n\'{u}mero de \'{o}rbitas peri\'{o}dicas de per\'{\i}odo $%
3 $, $4$ y $5$.
\end{enumerate}

\item \textbf{Modelo de crecimiento econ\'{o}mico de Solow en tiempo discreto%
}. Los ingredientes son:

\begin{enumerate}
\item \textit{Tecnolog\'{\i}a}: hay un solo bien $Y_{t}$ que se producen
usando dos factores de producci\'{o}n, capital $K_{t}$\ y trabajo $L_{t}$ de
acuerdo con la funci\'{o}n de producci\'{o}n $F\left( K,L\right) $ que
satisface las siguientes condiciones:

\begin{enumerate}
\item $F\left( \lambda K,\lambda L\right) =\lambda F\left( K,L\right) $; $%
\forall \lambda ,K,L\in 
%TCIMACRO{\U{211d} }%
%BeginExpansion
\mathbb{R}
%EndExpansion
^{+}$ (retornos constantes a escala)

\item $F\left( K,0\right) =F\left( 0,L\right) =0$; $\forall K,L\in 
%TCIMACRO{\U{211d} }%
%BeginExpansion
\mathbb{R}
%EndExpansion
^{+}$

\item $\frac{\partial F}{\partial K}>0,\frac{\partial F}{\partial L}>0,\frac{%
\partial ^{2}F}{\partial K^{2}}<0,\frac{\partial ^{2}F}{\partial L^{2}}<0$

\item $\lim\limits_{K\rightarrow 0}\frac{\partial F}{\partial K}%
=\lim\limits_{L\rightarrow 0}\frac{\partial F}{\partial L}=+\infty
;\lim\limits_{K\rightarrow +\infty }\frac{\partial F}{\partial K}%
=\lim\limits_{L\rightarrow +\infty }\frac{\partial F}{\partial L}=0$
(condiciones de Inada)
\end{enumerate}

\item \textit{Evoluci\'{o}n de los factores}: la fuerza de trabajo $L_{t}$
crece a la tasa constante $n$ de modo que%
\begin{equation}
L_{t+1}=\left( 1+n\right) L_{t}  \label{w}
\end{equation}%
y la tasa de crecimiento del stock de capital iguala la inversi\'{o}n neta $%
I=sF\left( K,L\right) $ menos la depreciaci\'{o}n del capital $\delta K:$%
\begin{equation}
K_{t+1}-K_{t}=sF\left( K_{t},L_{t}\right) -\delta K_{t}  \label{u}
\end{equation}

\item \textit{Evoluci\'{o}n de la econom\'{\i}a}: Si $k_{t}=\frac{K_{t}}{%
L_{t}}$ es el capital por trabajador y $f(k)=F\left( \frac{K}{L},1\right)
=F\left( k,1\right) $ es la funci\'{o}n de producci\'{o}n en forma intensiva
tenemos que: $f(0)=0$, $f^{\prime }(k)>0,\forall k\in 
%TCIMACRO{\U{211d} }%
%BeginExpansion
\mathbb{R}
%EndExpansion
^{+}$, $\lim\limits_{k\rightarrow +\infty }f^{\prime }(k)=0$, $%
\lim\limits_{k\rightarrow 0^{+}}f^{\prime }(k)=+\infty $ y $f^{\prime \prime
}(k)<0,\forall k\in 
%TCIMACRO{\U{211d} }%
%BeginExpansion
\mathbb{R}
%EndExpansion
^{+}$. Dividiendo la ecuaci\'{o}n (\ref{u}) por la ecuaci\'{o}n (\ref{w}) se
obtiene la ecuaci\'{o}n fundamental del modelo de Solow a tiempo discreto.
Esta ley describe como var\'{\i}a con el tiempo el capital per c\'{a}pita:%
\begin{equation*}
k_{t+1}=\frac{s}{1+n}f(k_{t})+\left( \frac{1-\delta }{1+n}\right) k_{t}
\end{equation*}%
Se pide estudiar cualitativamente la din\'{a}mica del modelo.
\end{enumerate}
\end{enumerate}

\newpage

\section{Soluciones:}

\begin{enumerate}
\item Expresi\'{o}n de $a_{n}$

\begin{enumerate}
\item $a_{n+1}=\left( 1+i\right) a_{n}-r;$ \ $a_{n}=\left( 1+i\right)
^{n}\lambda +\frac{r}{i}$

\item $a_{n+1}=a_{n}+n;a_{1}=0$

\item $a_{n+1}=a_{n}+a_{n-1};a_{2}=3$

\item $a_{n}=2$

\item $a_{n+1}=a_{n}+n$
\end{enumerate}

\item a) $a_{n+1}=\left( 1+r\right) a_{n}+T;$ \ b) $a_{n}=-\frac{T}{r}%
+\left( 1+r\right) ^{n}\left( a_{0}+\frac{T}{r}\right) $

\item $\underset{n\rightarrow +\infty }{\lim }x_{n}=\left\{ 
\begin{array}{c}
0\text{ \ si }-1<a<1 \\ 
\infty \text{ \ \ si }1<\left\vert a\right\vert%
\end{array}%
\right. ;$ $x_{n}=x_{0}$ $\forall n$ \ si $a=1$ \ y \ $x_{n}=\left(
-1\right) ^{n}x_{0}$ $\forall n$ \ si $a=-1$

\item Soluciones

\begin{enumerate}
\item $a_{n}=\left( \frac{3}{2}\right) ^{n}$

\item $a_{n}=\left( n-1\right) !a$

\item $a_{n}=\frac{1}{n}a_{1}$

\item $a_{n}=n+1$

\item $a_{n}=-3n^{2}-5n-8+11.2^{n}$

\item $a_{n}=-6+2^{n}$

\item $a_{n}=-2^{n-1}$
\end{enumerate}

\item Riccati

\begin{enumerate}
\item $y_{n+1}=-\frac{p}{q}y_{n}-\frac{1}{q};$ \ \ $x_{n}=\frac{1}{-\frac{1}{%
p+q}+\left( \frac{1}{x_{0}}+\frac{1}{p+q}\right) \left( -\frac{p}{q}\right)
^{n}}$

\item $y_{n+2}=\left( a+d\right) y_{n+1}+\left( bc-ad\right) y_{n}$

\item $x_{n}=$

\item $x_{n}=\left( x_{0}\right) ^{2^{n}};$ \ \ $\underset{n\rightarrow
+\infty }{\lim }x_{n}=\left\{ 
\begin{array}{c}
0\text{ \ si }-1<x_{0}<1 \\ 
\infty \text{ \ \ si }1<\left\vert x_{0}\right\vert%
\end{array}%
\right. $ \ y \ $x_{n}=1$ $\forall n$ \ si $x_{0}=\pm 1$

\item $x_{n}=\frac{1-\left( 1-2x_{0}\right) ^{2^{n}}}{2};$ \ \ $\underset{%
n\rightarrow +\infty }{\lim }x_{n}=\left\{ 
\begin{array}{c}
\frac{1}{2}\text{ \ si }0<x_{0}<1 \\ 
\infty \text{ \ \ si }1<\left\vert x_{0}\right\vert%
\end{array}%
\right. $ \ y \ $x_{n}=0$ $\forall n\geq 1$ \ si $x_{0}=0$ o $x_{0}=1$
\end{enumerate}

\item a) $p_{n+1}=\dfrac{2-p_{n}}{2};$ \ b) $p^{\ast }=\sqrt{3}-1$ \ atractor

\item a) $p_{n+1}=\dfrac{m_{D}-\lambda \left( m_{D}+m_{S}\right) }{m_{D}}%
p_{n}+\dfrac{\lambda \left( b_{D}-b_{S}\right) }{m_{D}};$ \ b) $p^{\ast }=%
\dfrac{b_{D}-b_{S}}{m_{S}+m_{D}}$

\item Si $x_{0}\leq 0\Rightarrow x_{n}\rightarrow 0$ y si $%
x_{0}>0\Rightarrow x_{n}\rightarrow +\infty $

\item a) $\underset{n\rightarrow +\infty }{\lim }x_{n}=+\infty $, \ $\forall
x_{0}$ \ b) $1$ atractor, $3$ repulsor \ c) $-1$ repulsor, $1$ punto silla

\item a) $f(x)-x>0;\forall x>-1,x\neq 1$, $f(x)-x<0;\forall x<-1$ y $f(1)=0;$%
\ \ b) $p^{\ast }=1,$ $f^{\prime }(1)=1$ y es un punto silla; c) $\underset{%
n\rightarrow +\infty }{\lim }x_{n}=1$; d) Si

\item a) $p^{\ast }=\frac{\alpha -1}{\beta };$ \ b) $f(p^{\ast })=\frac{1}{2}
$ \ atractor$;$ \ c) $p^{\ast }$ \ atractor $\forall \alpha >1;$ \ d) $x_{n}=%
\frac{1}{\frac{\alpha -1}{\beta }+\frac{1}{\alpha ^{n}}\lambda }$

\item puntos de equilibrio

\begin{enumerate}
\item $2$ repulsor y $3$ atractor

\item $0$ repulsor y $-2$ punto silla

\item $\frac{3}{4}$ repulsor y $\frac{1}{4}$ atractor

\item $\pm 1$ repulsores y $0$ atractor

\item $0$ repulsor

\item $-1$ repulsor y $0$ punto silla

\item $1$ repulsor y $0$ punto silla
\end{enumerate}

\item $x_{n}=\frac{1}{n}x_{0}\Rightarrow \underset{n\rightarrow +\infty }{%
\lim }x_{n}=0$

\item $x_{n}=\left( \underset{i=0}{\overset{n-1}{\prod }}a_{i}\right) x_{0}$
y de alli sale.

\item $f^{\prime }(1)f^{\prime }(-1)=0$ y de alli sale.

\item a) $\pm 2$ repulsores y $2$ atractor$;$ $\left\{ -1\pm \sqrt{2}%
\right\} $ $2$-ciclo repulsor\ b) $\left\{ 0,1\right\} $ $2$-ciclo atractor; 
$\frac{-1\pm \sqrt{5}}{2}$ puntos fijos

\item $\frac{1-\sqrt{17}}{4}<a<\frac{1+\sqrt{17}}{4};$ \ $b=a+1$

\item Una funci\'{o}n creciente no puede tener ciclos. Una decreciente si.

\item Mapa de Baker

\begin{enumerate}
\item La grafica de la funci\'{o}n $B(x)$ es\FRAME{dtbpF}{2.5394in}{2.5501in%
}{0pt}{}{}{Figure}{\special{language "Scientific Word";type
"GRAPHIC";maintain-aspect-ratio TRUE;display "USEDEF";valid_file "T";width
2.5394in;height 2.5501in;depth 0pt;original-width 2.3882in;original-height
2.3973in;cropleft "0";croptop "1";cropright "1";cropbottom "0";tempfilename
'PQP1HT04.wmf';tempfile-properties "XPR";}}

\item Sale de la definici\'{o}n: $B\left( x\right) =1\Leftrightarrow x=\frac{%
1}{4}$ o $x=\frac{3}{4}$ $B\left( x\right) =0\Leftrightarrow x=\frac{2}{4}$
y asi se sigue

\item El 2-ciclo es $\left\{ \dfrac{1}{3},\dfrac{2}{3}\right\} $ $\ $y es
repulsor. La grafica de la funci\'{o}n $B^{2}(x)$ es\FRAME{dtbpF}{2.5394in}{%
2.5567in}{0pt}{}{}{Figure}{\special{language "Scientific Word";type
"GRAPHIC";maintain-aspect-ratio TRUE;display "USEDEF";valid_file "T";width
2.5394in;height 2.5567in;depth 0pt;original-width 2.3882in;original-height
2.4039in;cropleft "0";croptop "1";cropright "1";cropbottom "0";tempfilename
'PQP1HT05.wmf';tempfile-properties "XPR";}}

\item Hay $2$ $3$-ciclos, $3$ $4$-ciclos y $6$ $5$-ciclos,
\end{enumerate}
\end{enumerate}

\end{document}
