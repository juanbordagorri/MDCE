\documentclass[11pt]{beamer}
\usepackage[utf8]{inputenc}
\usepackage[T1]{fontenc}
%\usepackage{natbib}
\usetheme{Pittsburgh}
\usepackage{verbatim} 
\usepackage[english]{babel}
\usepackage{epstopdf}
\usepackage{multicol}
%\titlegraphic{%\vspace*{1cm}
%	\includegraphics[width=2.5cm]{logo_udelar}
	%\hspace*{1cm}~%
%		\includegraphics[width=3.5cm]{logo_FCEA.png}
%}
\setbeamertemplate{navigation symbols}{}
\setbeamertemplate{footline}[frame number]
\AtBeginSection{ 
	\begin{frame}
		\frametitle{Index}
			\tableofcontents[currentsection]
	\end{frame}
}
\begin{document}
	\title{Modelos dinámicos y computacionales en Economía}
	\subtitle{Modelo de Oferta y Demanda Dinámica \\ (Modelo de la telaraña)}
	%\logo{}
	\institute{FCEA, UDELAR}
	\date{27 de agosto de 2024}
	%\subject{}
	%\setbeamercovered{transparent}
	%\setbeamertemplate{navigation symbols}{}
	\frame[plain]{\maketitle}
%\setbeamertemplate{background}{\includegraphics[width=2 cm]{logo_FCEA.png}}

\begin{frame}
\frametitle{Contenido de la clase:}
\begin{itemize}
\item Repaso de la clase pasada
\begin{enumerate}
    \item Ecuación en diferencias lineal de primer orden
\end{enumerate}
	\item Modelo de la telaraña
	\begin{enumerate}
	\item Modelo original 
	\item Modelo con expectativas adaptativas
\end{enumerate}		
	\item Otras extensiones
	\begin{enumerate}
		\item Modelo con inventario 
	\end{enumerate}	
\item Solución numérica en R
	\begin{enumerate}
	\item ¿Cómo escribimos un modelo en R?
	\item Modelo original
	\item Extensiones
%	\item Funciones en R
\end{enumerate}	
\end{itemize}
\end{frame}

\begin{frame}
\frametitle{Repaso de la clase pasada}
\framesubtitle{Ecuación en diferencias lineal de primer orden}
\small
\[ \boxed{\ \ (E) \ \ x_{n+1}=\;a_n\; x_{n}+ b_n \ \ } \]

    \begin{itemize}
        \item Solución de la ecuación homogénea: con condici\'{o}n inicial $x_0=c$ viene dada por:
\[ x_n \ = \ \left(\prod_{i=0}^{i=n-1} a_i \right) \ c =  a_{n-1}\ .\ .\ .\ .\ a_2\ a_1\ a_0 \ c\ \ , \ \ \forall\ n \geq 1. \]
        \item Solución general: 
    \end{itemize}
\[ \boxed{\ \ sol.\ general\ de\ (E) \ = \  sol.\ general\ de\ (H) \ + \ sol.\ particular\ de\ (E) \ \ } \]
Solución particular?
\begin{itemize}
    \item $X_n=A \longrightarrow X_{n+1}= A$
    \item $X_n=A n + B\longrightarrow X_{n+1}= A (n+1) + B$
    \item $X_n=A n^2 \longrightarrow X_{n+1}= A (n+1)^2 $; ....
\end{itemize}
\end{frame}

\begin{frame}
\frametitle{Repaso de la clase pasada}
\framesubtitle{Ecuación en diferencias lineal de primer orden}
    
{\bf    Caso especial: coeficientes constantes}
    \[ \boxed{\ \ (E) \ \ x_{n+1}=\;a\; x_{n}+ b \ \ } \]
\begin{itemize}
        \item Solución de la ecuación homogénea: $c\ a^n$
        \item Solución general: Si $a\not=1$ entonces la soluci\'{o}n de $(E)$ con condici\'{o}n inicial $x_0$ es
\[ x_n=\dfrac{b}{1-a}+ \left( x_0 - \dfrac{b}{1-a} \right)a^n \ \ , \ \ \forall \  n \geq 0. \]
Adem\'{a}s, en este caso la \'{u}nica soluci\'{o}n constante (punto de equilibrio) es $\dfrac{b}{1-a}$.
    \end{itemize}
    
\end{frame}

\begin{frame}
\frametitle{Cobweb model: introducción}
\begin{itemize}
\item Introducido por Kaldor (1934)\footnote{Kaldor, N. (1934). A classificatory note on the determinateness of equilibrium. \textit{The Review of Economic Studies, 1}(2), 122-136.} y Ezekiel (1938)\footnote{Ezekiel, M. (1938). The cobweb theorem. \textit{The Quarterly Journal of Economics, 52}(2), 255-280.}. 
	\item Este modelo muestra que los precios pueden estar sujetos a fluctuaciones periódicas.
	\item Importancia del tiempo discreto.
	\item Para algunos valores de los parámetros, se alcanza el equilibrio.
	\item El comportamiento dinámico de los agentes \underline{no siempre} converge al equilibrio.
\end{itemize}
\end{frame}

\begin{frame}
\frametitle{Cobweb model: modelo original}
\framesubtitle{Presentación del modelo}
\begin{multicols}{2}
\underline{Ecuaciones:}
\begin{enumerate}
	\item $D_t=\alpha-\beta p_t$
	\item $S_t=-\gamma+\delta p_{t-1}$
	\item $D_t=S_t$ (condición de equilibrio)
\end{enumerate}
%\vspace{1.5cm}
\underline{Variables:}
\begin{itemize}
	\item D: cantidad demandada
	\item S: cantidad ofrecida
	\item p: precio
\end{itemize}
\vspace{1cm}
\underline{Parámetros: }
\begin{itemize}
	\item $\alpha, \beta$ (parámetros demanda)
	\item $\gamma, \delta$: (parámetros oferta)
\end{itemize}
\underline{Dinámica:}
\begin{equation}
p_{t+1}=-\dfrac{\delta}{\beta}p_t+\dfrac{\alpha+\gamma}{\beta}
\end{equation}	
\end{multicols}
\end{frame}

\begin{frame}
\frametitle{Cobweb model: modelo original}
\framesubtitle{Solución del modelo}	
Solución particular:
\begin{equation}\label{solpart1}
p^{*}=\dfrac{\alpha+\gamma}{\beta+\delta}
\end{equation}
Solución general:
\begin{equation}
p_t=\left( p_1-\dfrac{\alpha+\gamma}{\beta+\delta}\right) \left(-\dfrac{\delta}{\beta} \right)^{t-1}+
\dfrac{\alpha+\gamma}{\beta+\delta} 
\end{equation}
Importante:
\begin{itemize}
\scriptsize	\item La ecuación \ref{solpart1} refleja el equilibrio intertemporal del modelo.
	\item $\left( p_1-\dfrac{\alpha+\gamma}{\beta+\delta}\right)$ nos muestra si comenzamos por encima o por debajo del precio de equilibrio y la distancia a él.
	\item La expresión $\left|\dfrac{\delta}{\beta} \right|$ es importante para conocer la estabilidad del modelo: atractor si es < 1, repulsor si es > 1, oscilatorio (puntos períodicos de período 2) si es igual a 1.
\end{itemize}
\end{frame}

\begin{frame}
	\frametitle{Cobweb model: modelo con expectativas adaptativas}
	\framesubtitle{Presentación del modelo}
	\begin{multicols}{2}
		\underline{Ecuaciones:}
		\begin{enumerate}
			\item $D_t=\alpha-\beta p_t$
			\item $S_t=-\gamma+\delta p^{e}_{t}$
			\item $p^{e}_{t}=\lambda p_{t-1}+(1-\lambda) p^{e}_{t-1}$
			\item $D_t=S_t$ (condición de equilibrio)
		\end{enumerate}
		%\vspace{1.5cm}
		\underline{Variables:}
		\begin{itemize}
			\item D: cantidad demandada
			\item S: cantidad ofrecida
			\item p: precio
			\item $p^{e}$: precio esperado
		\end{itemize}
		\vspace{1cm}
		\underline{Parámetros: }
		\begin{itemize}
			\item $\alpha, \beta$ (parámetros demanda)
			\item $\gamma, \delta$ (parámetros oferta)
			\item $\lambda$: parámetro expectativas, $\lambda \in [0,1]$
		\end{itemize}
		\underline{Dinámica:}
		\begin{equation}
		p_{t+1}=\left[ \dfrac{\beta-\lambda(\beta+\delta)}{\beta}\right] p_t+\dfrac{\alpha+\gamma}{\beta}
		\end{equation}	
	\end{multicols}
\end{frame}

\begin{frame}
	\frametitle{Cobweb model: modelo con expectativas adaptativas}
	\framesubtitle{Solución del modelo}	
	Solución particular:
	\begin{equation}\label{solpart2}
	p^{*}=\dfrac{\alpha+\gamma}{\beta+\delta}
	\end{equation}
	Solución general:
	\begin{equation}
	p_t=\left( p_1-\dfrac{\alpha+\gamma}{\beta+\delta}\right) \left[ \dfrac{\beta-\lambda(\beta+\delta)}{\beta}\right]^{t-1}+
	\dfrac{\alpha+\gamma}{\beta+\delta} 
	\end{equation}
	Importante:
	\begin{itemize}
		\scriptsize	\item La ecuación \ref{solpart2} refleja el equilibrio intertemporal del modelo.
		\item $\left( p_1-\dfrac{\alpha+\gamma}{\beta+\delta}\right)$ nos muestra si comenzamos por encima o por debajo del precio de equilibrio y la distancia a él.
		\item La expresión $\left[ \dfrac{\beta-\lambda(\beta+\delta)}{\beta}\right]$ es importante para conocer la estabilidad del modelo.
	\end{itemize}
\end{frame}

\begin{frame}
	\frametitle{Cobweb model: modelo con inventario}
\framesubtitle{Introducción}	
	\begin{itemize}
		\item En este caso, levantamos el supuesto que $D_t=S_t$, $\forall t$.
		\item Como la oferta reacciona a las variaciones de precios, los productores guardan un inventario para hacer frente a imprevistos.
	\end{itemize}
\end{frame}

\begin{frame}
\frametitle{Cobweb model: modelo con inventario}
\framesubtitle{Presentación del modelo}
\begin{multicols}{2}
\underline{Ecuaciones:}
\begin{enumerate}
\item $D_t=\alpha-\beta p_t$
\item $S_t=-\gamma+\delta p_{t}$
\item $p_{t+1}=p_{t}-\sigma(S_t-D_t)$ \end{enumerate}
		%\vspace{1.5cm}
\underline{Variables:}
\begin{itemize}
\item D: cantidad demandada
\item S: cantidad ofrecida
\item p: precio
\end{itemize}
\vspace{1cm}
\underline{Parámetros: }
\begin{itemize}
\item $\alpha, \beta$ (parámetros demanda)
\item $\gamma, \delta$: (parámetros oferta)
\item $\sigma$: parámetro de ajuste de precios
\end{itemize}	
\end{multicols}
\end{frame}


\begin{frame}
\frametitle{Cobweb model: modelo con inventario}
\framesubtitle{Dinámica del modelo}
\underline{Pasos a seguir:}
\begin{itemize}
	\item sustituimos $S_t$ y $D_t$ en la tercera ecuación para que dependa sólo del precio y los parámetros.
\end{itemize}
\begin{equation}
p_{t+1}=[1-\sigma(\beta+\delta)]p_t+\sigma(\alpha+\gamma)
\end{equation}
\begin{enumerate}
	\item ¿Solución particular?
	\item ¿Solución general?
\end{enumerate}
\end{frame}

\begin{frame}
	\frametitle{Cobweb model: modelo con inventario}
	\framesubtitle{Solución particular y general del modelo}
	\begin{enumerate}
		\item Solución particular: 	\begin{equation}\label{solpart3}
		p^{*}=\dfrac{\alpha+\gamma}{\beta+\delta}
		\end{equation}
		\item Solución general:
	\begin{equation}
p_t=\left( p_1-\dfrac{\alpha+\gamma}{\beta+\delta}\right) [1-\sigma(\beta+\delta)]^{t-1}+
\dfrac{\alpha+\gamma}{\beta+\delta} 
\end{equation}

	\end{enumerate}
Importante:
\begin{itemize}
	\scriptsize	\item La ecuación \ref{solpart3} refleja (el mismo) el equilibrio intertemporal del modelo.
	\item $\left( p_1-\dfrac{\alpha+\gamma}{\beta+\delta}\right)$ nos muestra (nuevamente) si comenzamos por encima o por debajo del precio de equilibrio y la distancia a él.
	\item Estabilidad? Ver valores de $\sigma$ para los cuales la solución es estable.\\
	\onslide<2->{\textcolor{blue}{Solución: $0<\sigma<\dfrac{2}{\beta+\delta}$}}
\end{itemize}
\end{frame}

\begin{frame}
\frametitle{Escribir un modelo en R}
\framesubtitle{¿Qué datos necesitamos?}
Para escribir un modelo, necesitamos conocer:
\begin{itemize}
	\item Parámetros del modelo.
	\item Variables del modelo.
	\item Manera en que se relacionan los parámetros y las variables (ecuaciones).
	\item Valores iniciales de las variables.
\end{itemize}
\end{frame}

\begin{frame}
	\frametitle{Escribir un modelo en R}
	\framesubtitle{Ejemplo 1: modelo Cobweb}
\begin{itemize}
	\item Dar valores a los parámetros:
	\begin{itemize}
		\item $\beta=10$; \hspace{0.55cm} $\delta=5$;
		\item $\alpha=1000$; $-\gamma=250$.
	\end{itemize}
	\item Generar las variables relevantes para la dinámica (p, S, D).
	\item Establecer una condición inicial para p: $p_{(t=1)}=25$
	\item Escribir la estructura de control para generar la dinámica del sistema:
	\begin{itemize}
		\item utilizamos la ecuación de movimiento del sistema.
		\item por 50 iteraciones (utilizar un $for$ o un $while$)
		\item Para los valores iniciales, ¿qué podemos decir de la estabilidad del modelo? \onslide<2->{Probar con:
			\begin{itemize}
				\item $\beta=\delta=10$
				\item $\beta=10$, $\delta=10.5$
			\end{itemize}}
	\end{itemize}
\end{itemize}
\end{frame}

\begin{frame}
	\frametitle{Escribir un modelo en R}
\framesubtitle{Ejemplo 2: modelo Cobweb con expectativas adaptativas}
\begin{itemize}
	\item Utilizaremos los mismos valores de los parámetros $\longrightarrow \lambda=0.5$  
	\item ¿qué debemos cambiar en el código anterior para trabajar con expectativas adaptativas?
	\item Mostrar que el equilibrio es el mismo $\longrightarrow$ ¿qué es lo que cambia?
\end{itemize}
\end{frame}

\begin{frame}
	\frametitle{Escribir un modelo en R}
	\framesubtitle{Ejemplo 3: modelo Cobweb con inventario}
	\begin{itemize}
		\item Utilizaremos los mismos valores de los parámetros $\longrightarrow \sigma=0.5$  
		\item ¿qué debemos cambiar en el código original para trabajar con inventarios?
		\item Mostrar que el equilibrio es el mismo $\longrightarrow$ ¿qué es lo que cambia?
	\end{itemize}
\end{frame}

\end{document}

